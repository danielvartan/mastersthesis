%:::% class attribute begin/end %:::%

% -----
% Cover (mandatory)
% -----

%:::% cover begin %:::%
\imprimircapa
%:::% cover end %:::%

% -----
% Title page (mandatory)
% -----

%:::% approval-sheet begin %:::%
\imprimirfolhaderosto
%:::% approval-sheet end %:::%

% -----
% Cataloging record (mandatory)
% -----

%:::% cataloging-record begin %:::%
\begin{fichacatalografica}
\hyphenpenalty=100000
%:::% cataloging-record body begin %:::%
I authorize the full or partial reproduction of this work by any conventional or electronic means for the purposes of study and research, provided that the source is cited.
\small


\vfill
\begin{center}
CATALOGUING IN PUBLICATION

(University of São Paulo. School of Arts, Sciences and Humanities. Library)

{\tiny CRB 8-4936}

\medskip
\ABNTEXfontereduzida
\setlength{\fboxsep}{1cm}
\fbox{
\begin{minipage}[c][6cm]{12cm}
Azevedo, Daniel Kachvartanian de

\hspace{0.5cm} {\imprimirtitulo}  / {\imprimirautor} ; supervisor, {\imprimirorientador}. -- {\imprimirdata}

\hspace{0.5cm} {\thelastpage} p : il.

\smallskip
\hspace{0.5cm} {\imprimirtipotrabalho} (\imprimirtituloacademico) -- {\imprimirprograma}, {\imprimirescola}, {\imprimiruniversidade}.

\hspace{0.5cm} {\imprimirnotadeversao}.

\smallskip
\hspace{0.5cm} 1. Complexity science. 2. Chronobiology. 3. Biological rhythms. 4. Chronotypes. 5. Circadian phenotypes. 6. Entrainment. 7. Latitude. I. Rodrigues Neto, Camilo, supervisor II. Title.

\end{minipage}
}
\end{center}
\vspace{\hugeskipamount}
%:::% cataloging-record body end %:::%
\end{fichacatalografica}
%:::% cataloging-record end %:::%

% -----
% Approval sheet (mandatory)
% -----

%:::% approval-sheet begin %:::%
\begin{folhadeaprovacao}[\folhadeaprovacaoname]
\hyphenpenalty=100000
%:::% approval-sheet body begin %:::%
{\imprimirtipotrabalho} by {\imprimirautor}, under the title \textbf{\imprimirtitulo}, presented to the {\imprimirescola} at the {\imprimiruniversidade}, as a requirement for the degree of {\imprimirtituloacademico} by the {\imprimirprograma}, in the concentration area of {\imprimirareadeconcentracao}.

\vspace{\hugeskipamount}
Approved on \ \underline{\hspace{5cm}} \ , \underline{\hspace{2.5cm}} .

\vspace{\hugeskipamount}
\begin{center}
  Examination committee
\end{center}

\vspace{\smallskipamount}
Committee chair:

\vspace{\tinyskipamount}
\begingroup

\AtBeginEnvironment{tabular}{
  \normalsize
  \renewcommand{\arraystretch}{2}
}

\setlength{\arrayrulewidth}{0pt}
\setlength{\tabcolsep}{0cm}
\begin{tabular}{m{2.5cm} P{13.5cm}}
  Prof. Dr. & \hrulefill \\
  Institution & \hrulefill \\
\end{tabular}

\vspace{\bigskipamount}
Examiners:

\vspace{\tinyskipamount}
\begin{tabular}{m{2.5cm} P{13.5cm}}
  Prof. Dr. & \hrulefill \\
  Institution & \hrulefill \\
  Evaluation & \hrulefill \\
\end{tabular}

\vspace{\smallskipamount}
\begin{tabular}{m{2.5cm} P{13.5cm}}
  Prof. Dr. & \hrulefill \\
  Institution & \hrulefill \\
  Evaluation & \hrulefill \\
\end{tabular}

\vspace{\smallskipamount}
\begin{tabular}{m{2.5cm} P{13.5cm}}
  Prof. Dr. & \hrulefill \\
  Institution & \hrulefill \\
  Evaluation & \hrulefill \\
\end{tabular}
\endgroup
%:::% approval-sheet body end %:::%
\end{folhadeaprovacao}
%:::% approval-sheet end %:::%

% -----
% Acknowledgments (optional)
% -----

%:::% acknowledgments begin %:::%
\begin{agradecimentos}[\agradecimentosname]
\hyphenpenalty=100000
%:::% acknowledgments body begin %:::%

This work would not have been possible without the support, love, and
guidance of many remarkable individuals and organizations. I extend my
deepest gratitude to:

\smallskip

My beloved partner, Salete Perroni (Sal), whose unwavering support has
been my constant source of strength. My mother, for her unconditional
love, and my sister and brother, for their companionship throughout
life's journey.

My scientific collaborators and dear friends, Alicia Rafaelly Vilefort
Sales and Maria Augusta Medeiros de Andrade. To Professor Humberto
Miguel Garay Malpartida, for his steadfast support, unwavering
principles, and integrity, which shone through when it was most needed.

My supervisor, Professor Camilo Rodrigues Neto, who introduced me to
complexity science in 2012, guided my dissertation with patience, and
demonstrated exceptional virtue in mediating a challenging supervisory
transition. I am equally grateful to Professor Carlos Molina Mendes, who
handled this transition with remarkable speed, impartiality, and
professionalism.

My cherished friends: Alex Azevedo Martins, Augusto Amado, Carina
(Cacau) Prado, Ítalo Alves Bezerra do Nascimento, Júlia Mafra, Letícia
Nery de Figueiredo, Marcelo Ricardo Fernandes Roschel, Reginaldo Noveli,
Sílvia Capelanes, and Vanessa Simon Silva.

Finally, this journey would not have been possible without the
institutional support of USP's Support Program for Student Permanence
and Education (PAPFE) and the Coordination for the Improvement of Higher
Education Personnel (CAPES), whose contributions have been invaluable.

\smallskip
\begingroup
\renewcommand{\baselinestretch}{1}

\noindent This study was financed in part by the Coordenação de
Aperfeiçoamento de Pessoal de Nível Superior - Brasil
(\href{https://www.gov.br/capes/}{CAPES}) - Finance Code 001, Grant
number 88887.703720/2022-00.

\endgroup

%:::% acknowledgments body end %:::%
\end{agradecimentos}
%:::% acknowledgments end %:::%

% -----
% Epigraph (optional)
% -----

%:::% epigraph begin %:::%
\begin{epigrafe}[] % \epigraphname | Keep #1 empty.
\vspace*{\fill} % Don't change it.
\begin{flushright}
%:::% epigraph body begin %:::%
\textit{Nullius in verba}\footnotemark{}

\footnotetext{
  The Royal Society. (n.d.). \textit{History of the Royal Society}. \href{https://royalsociety.org/about-us/history}{https://royalsociety.org/about-us/history}
}
%:::% epigraph body end %:::%
\end{flushright}
\end{epigrafe}
%:::% epigraph end %:::%

% -----
% Abstract in the vernacular language (mandatory)
% -----

%:::% vernacular-abstract begin %:::%
\begin{resumoenv}[\resumoname]
 %:::% vernacular-abstract reference begin %:::%
Vartanian, D. ({\imprimirdata}). \textit{\imprimirtitulo} [{\imprimirtipodetituloacademico}'s {\imprimirtipotrabalho}, {\imprimiruniversidade}].
%:::% vernacular-abstract reference end %:::%

%:::% vernacular-abstract body begin %:::%

Although much is known about circadian rhythms, further research with
larger samples is needed to gain a better understanding of the
expression of temporal phenotypes. This thesis investigates the
hypothesis that latitude influences chronotype expression, based on the
premise that polar regions receive less sunlight annually than
equatorial regions. This difference in photoperiod suggests a stronger
solar zeitgeber in equatorial areas, potentially leading to greater
entrainment of circadian rhythms with the light/dark cycle. This, in
turn, could reduce the amplitude and diversity of circadian phenotypes,
resulting in a higher prevalence of morningness in equatorial
populations. To test this hypothesis, data from \(65.824\) individuals
across Brazil, collected in 2017 based on the Munich ChronoType
Questionnaire (MCTQ), were analyzed using nested linear regression
models. The analysis revealed a negligible effect of latitude on
chronotype (Cohen's \(f^2 = 0.0030818242\),
\(95\% \ \text{IC}[0, 0.0121371208]\)), a finding that contrasts with
recent studies. Although the hypothesis is plausible and consistent with
evolutionary theories of temporal biological systems, these results
suggest that the phenomenon of entrainment is more complex than
previously assumed.

%:::% vernacular-abstract body end %:::%

%:::% vernacular-abstract keywords begin %:::%
\hyphenpenalty=100000
\begin{tabular}{p{2.3cm} p{13.6cm}}
  \textbf{Keywords}: & Complexity science. Complex systems. Chronobiology. Biological rhythms. Chronotypes. Circadian phenotypes. Sleep. Entrainment. Latitude. MCTQ.
\end{tabular}
%:::% vernacular-abstract keywords end %:::%
\end{resumoenv}
%:::% vernacular-abstract end %:::%

% -----
% Abstract in the foreign language (mandatory)
% -----

%:::% foreign-abstract begin %:::%
\begin{resumoenv}[\resumoestrangeironame]
\begin{otherlanguage*}{brazil}
%:::% foreign-abstract reference begin %:::%
Vartanian, D. ({\imprimirdata}). \textit{A latitude está associada ao cronotipo?} [Dissertação de Mestrado, Universidade de São Paulo].
%:::% foreign-abstract reference end %:::%

%:::% foreign-abstract body begin %:::%

Embora muito se saiba sobre ritmos circadianos, mais pesquisas com
amostras maiores são necessárias para obter uma melhor compreensão da
expressão de fenótipos temporais. Esta dissertação investiga a hipótese
de que a latitude influencia a expressão do cronotipo, com base na
premissa de que as regiões polares recebem menos luz solar anualmente do
que as regiões equatoriais. Essa diferença no fotoperíodo sugere um
\emph{zeitgeber} solar mais forte nas áreas equatoriais, levando
potencialmente a um maior \emph{entrainment} dos ritmos circadianos com
o ciclo claro-escuro. Isso, por sua vez, poderia reduzir a amplitude e a
diversidade dos fenótipos circadianos, resultando em uma maior
prevalência de matutinidade em populações equatoriais. Para testar essa
hipótese, dados de \(65.824\) indivíduos em todo o Brasil, coletados em
2017 com base no Questionário de Cronotipo de Munique (MCTQ), foram
analisados usando modelos de regressão linear aninhados. A análise
revelou um efeito negligenciável da latitude no cronotipo (\(f^2\) de
Cohen \(= 0,0030818242\), \(95\% \ \text{IC}[0; 0,0121371208]\)), um
achado que contrasta com estudos recentes. Embora a hipótese seja
plausível e consistente com as teorias evolutivas dos sistemas
biológicos temporais, esses resultados sugerem que o fenômeno do
\emph{entrainment} é mais complexo do que se imagina.

%:::% foreign-abstract body end %:::%

%:::% foreign-abstract keywords begin %:::%
\hyphenpenalty=100000
\begin{tabular}{p{3.6cm} p{12.3cm}}
  \textbf{Palavras-chaves}: &  Ciência da complexidade. Sistemas complexos. Cronobiologia. Ritmos biológicos. Cronotipos. Fenótipos circadianos. Sono. Entrainment. Latitude. MCTQ.
\end{tabular}
%:::% foreign-abstract keywords end %:::%
\end{otherlanguage*}
\end{resumoenv}
%:::% foreign-abstract end %:::%

% -----
% Table of contents (mandatory)
% -----

%:::% table-of-contents begin %:::%
\pdfbookmark[0]{\contentsname}{toc}
\tableofcontents*
\cleardoublepage
%:::% table-of-contents end %:::%

% -----
% Other additions
% -----

%:::% other-before-body begin %:::%
%:::% other-before-body end %:::%
