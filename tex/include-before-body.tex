%:::% class attribute begin/end %:::%

% -----
% Cover (mandatory)
% -----

%:::% cover begin %:::%
\imprimircapa
%:::% cover end %:::%

% -----
% Title page (mandatory)
% -----

%:::% approval-sheet begin %:::%
\imprimirfolhaderosto
%:::% approval-sheet end %:::%

% -----
% Cataloging record (mandatory)
% -----

%:::% cataloging-record begin %:::%
\begin{fichacatalografica}
\hyphenpenalty=100000
%:::% cataloging-record body begin %:::%
I authorize the full or partial reproduction of this work by any conventional or electronic means for the purposes of study and research, provided that the source is cited.
\small


\vfill
\begin{center}
CATALOGUING IN PUBLICATION

(University of São Paulo. School of Arts, Sciences and Humanities. Library)

{\tiny CRB 8-4936}

\medskip
\ABNTEXfontereduzida
\setlength{\fboxsep}{1cm}
\fbox{
\begin{minipage}[c][6cm]{12cm}
Azevedo, Daniel Kachvartanian de

\hspace{0.5cm} {\imprimirtitulo}  / {\imprimirautor} ; supervisor, {\imprimirorientador}. -- {\imprimirdata}

\hspace{0.5cm} {\thelastpage} p : il.

\smallskip
\hspace{0.5cm} {\imprimirtipotrabalho} (\imprimirtituloacademico) -- {\imprimirprograma}, {\imprimirescola}, {\imprimiruniversidade}.

\hspace{0.5cm} {\imprimirnotadeversao}.

\smallskip
\hspace{0.5cm} 1. Complexity science. 2. Chronobiology. 3. Biological rhythms. 4. Chronotypes. 5. Circadian phenotypes. 6. Entrainment. 7. Latitude. I. Rodrigues Neto, Camilo, supervisor II. Title.

\end{minipage}
}
\end{center}
\vspace{\hugeskipamount}
%:::% cataloging-record body end %:::%
\end{fichacatalografica}
%:::% cataloging-record end %:::%

% -----
% Approval sheet (mandatory)
% -----

%:::% approval-sheet begin %:::%
\begin{folhadeaprovacao}[\folhadeaprovacaoname]
\hyphenpenalty=100000
%:::% approval-sheet body begin %:::%
{\imprimirtipotrabalho} by {\imprimirautor}, under the title \textbf{\imprimirtitulo}, presented to the {\imprimirescola} at the {\imprimiruniversidade}, as a requirement for the degree of {\imprimirtituloacademico} by the {\imprimirprograma}, in the concentration area of {\imprimirareadeconcentracao}.

\vspace{\hugeskipamount}
Approved on \_\_\_\_\_\_\_\_\_\_\_\_\_\_\_\_\_\_\_\_ , \_\_\_\_\_\_\_\_\_\_ .

\vspace{\hugeskipamount}
\begin{center}
  Examination committee
\end{center}

\vspace{\smallskipamount}
Committee chair:

\vspace{\tinyskipamount}
\begingroup

\AtBeginEnvironment{tabular}{
  \normalsize
  \renewcommand{\arraystretch}{2}
}

\setlength{\arrayrulewidth}{0pt}
\setlength{\tabcolsep}{0cm}
\begin{tabular}{m{2cm} P{14cm}}
  Prof. Dr. & \_\_\_\_\_\_\_\_\_\_\_\_\_\_\_\_\_\_\_\_\_\_\_\_\_\_\_\_\_\_\_\_\_\_\_\_\_\_\_\_\_\_\_\_\_\_\_\_\_\_\_\_\_\_\_ \\
  Institution & \_\_\_\_\_\_\_\_\_\_\_\_\_\_\_\_\_\_\_\_\_\_\_\_\_\_\_\_\_\_\_\_\_\_\_\_\_\_\_\_\_\_\_\_\_\_\_\_\_\_\_\_\_\_\_ \\
\end{tabular}

\vspace{\bigskipamount}
Examiners:

\vspace{\tinyskipamount}
\begin{tabular}{m{2cm} P{14cm}}
  Prof. Dr. & \_\_\_\_\_\_\_\_\_\_\_\_\_\_\_\_\_\_\_\_\_\_\_\_\_\_\_\_\_\_\_\_\_\_\_\_\_\_\_\_\_\_\_\_\_\_\_\_\_\_\_\_\_\_\_ \\
  Institution & \_\_\_\_\_\_\_\_\_\_\_\_\_\_\_\_\_\_\_\_\_\_\_\_\_\_\_\_\_\_\_\_\_\_\_\_\_\_\_\_\_\_\_\_\_\_\_\_\_\_\_\_\_\_\_ \\
  Evaluation & \_\_\_\_\_\_\_\_\_\_\_\_\_\_\_\_\_\_\_\_\_\_\_\_\_\_\_\_\_\_\_\_\_\_\_\_\_\_\_\_\_\_\_\_\_\_\_\_\_\_\_\_\_\_\_ \\
\end{tabular}

\vspace{\smallskipamount}
\begin{tabular}{m{2cm} P{14cm}}
  Prof. Dr. & \_\_\_\_\_\_\_\_\_\_\_\_\_\_\_\_\_\_\_\_\_\_\_\_\_\_\_\_\_\_\_\_\_\_\_\_\_\_\_\_\_\_\_\_\_\_\_\_\_\_\_\_\_\_\_ \\
  Institution & \_\_\_\_\_\_\_\_\_\_\_\_\_\_\_\_\_\_\_\_\_\_\_\_\_\_\_\_\_\_\_\_\_\_\_\_\_\_\_\_\_\_\_\_\_\_\_\_\_\_\_\_\_\_\_ \\
  Evaluation & \_\_\_\_\_\_\_\_\_\_\_\_\_\_\_\_\_\_\_\_\_\_\_\_\_\_\_\_\_\_\_\_\_\_\_\_\_\_\_\_\_\_\_\_\_\_\_\_\_\_\_\_\_\_\_ \\
\end{tabular}

\vspace{\smallskipamount}
\begin{tabular}{m{2cm} P{14cm}}
  Prof. Dr. & \_\_\_\_\_\_\_\_\_\_\_\_\_\_\_\_\_\_\_\_\_\_\_\_\_\_\_\_\_\_\_\_\_\_\_\_\_\_\_\_\_\_\_\_\_\_\_\_\_\_\_\_\_\_\_ \\
  Institution & \_\_\_\_\_\_\_\_\_\_\_\_\_\_\_\_\_\_\_\_\_\_\_\_\_\_\_\_\_\_\_\_\_\_\_\_\_\_\_\_\_\_\_\_\_\_\_\_\_\_\_\_\_\_\_ \\
  Evaluation & \_\_\_\_\_\_\_\_\_\_\_\_\_\_\_\_\_\_\_\_\_\_\_\_\_\_\_\_\_\_\_\_\_\_\_\_\_\_\_\_\_\_\_\_\_\_\_\_\_\_\_\_\_\_\_ \\
\end{tabular}
\endgroup
%:::% approval-sheet body end %:::%
\end{folhadeaprovacao}
%:::% approval-sheet end %:::%

% -----
% Acknowledgments (optional)
% -----

%:::% acknowledgments begin %:::%
\begin{agradecimentos}[\agradecimentosname]
\hyphenpenalty=100000
%:::% acknowledgments body begin %:::%

This work would not have been possible without the support, love, and
guidance of many remarkable individuals and organizations. I extend my
deepest gratitude to:

\smallskip

My beloved partner, Salete Perroni (Sal), whose unwavering support has
been my constant source of strength. My mother, for her unconditional
love, and my sister and brother, for their companionship throughout
life's journey.

My scientific collaborators and dear friends, Alicia Rafaelly Vilefort
Sales and Maria Augusta Medeiros de Andrade. Professor Humberto Miguel
Garay Malpartida, for his steadfast support, unwavering principles, and
integrity, which shone through when it was most needed.

My supervisor, Professor Camilo Rodrigues Neto, who introduced me to
complexity science in 2012, guided my dissertation with patience, and
demonstrated exceptional virtue in mediating a challenging supervisory
transition. Professor Carlos Molina Mendes, who handled this transition
with remarkable speed, impartiality, and professionalism.

My cherished friends: Alex Azevedo Martins, Augusto Amado, Carina
(Cacau) Prado, Ítalo Alves Bezerra do Nascimento, Júlia Mafra, Letícia
Nery de Figueiredo, Marcelo Ricardo Fernandes Roschel, Reginaldo Noveli,
Sílvia Capelanes, and Vanessa Simon Silva.

Finally, this journey was made possible by the institutional support of
USP's Support Program for Student Permanence and Education (PAPFE) and
the Coordination for the Improvement of Higher Education Personnel
(CAPES), whose contributions have been invaluable.

\smallskip
\begingroup
\renewcommand{\baselinestretch}{1}

\noindent This study was financed in part by the Coordenação de
Aperfeiçoamento de Pessoal de Nível Superior - Brasil
(\href{https://www.gov.br/capes/}{CAPES}) - Finance Code 001, Grant
number 88887.703720/2022-00.

\endgroup

%:::% acknowledgments body end %:::%
\end{agradecimentos}
%:::% acknowledgments end %:::%

% -----
% Epigraph (optional)
% -----

%:::% epigraph begin %:::%
\begin{epigrafe}[] % \epigraphname | Keep #1 empty.
\vspace*{\fill} % Don't change it.
\begin{flushright}
%:::% epigraph body begin %:::%
\textit{Nullius in verba}\footnotemark{}

\footnotetext{
  The Royal Society. (n.d.). \textit{History of the Royal Society}. \href{https://royalsociety.org/about-us/history}{https://royalsociety.org/about-us/history}
}
%:::% epigraph body end %:::%
\end{flushright}
\end{epigrafe}
%:::% epigraph end %:::%

% -----
% Abstract in the vernacular language (mandatory)
% -----

%:::% vernacular-abstract begin %:::%
\begin{resumoenv}[\resumoname]
 %:::% vernacular-abstract reference begin %:::%
Vartanian, D. ({\imprimirdata}). \textit{\imprimirtitulo} [{\imprimirtipodetituloacademico}'s {\imprimirtipotrabalho}, {\imprimiruniversidade}].
%:::% vernacular-abstract reference end %:::%

%:::% vernacular-abstract body begin %:::%

Theories on circadian rhythms are well-established in science, but there
is still a need to test them in larger samples to gain a better
understanding of the expression of temporal phenotypes. This thesis
investigates the hypothesis that latitude influences chronotype
expression, based on the idea that regions closer to the poles receive
less sunlight over the year than equatorial regions. This difference
suggests that equatorial areas have a stronger solar zeitgeber, which
could lead to greater synchronization of circadian rhythms with the
light-dark cycle, reducing the amplitude and diversity of circadian
phenotypes, resulting in a higher propensity for morningness in those
populations. To test this hypothesis, data from \(65,824\) individuals
from all regions of Brazil were analyzed, collected in 2017 based on the
Munich ChronoType Questionnaire (MCTQ). The analysis, using nested
linear regression models, revealed a negligible effect of latitude on
the variation in chronotype expression (Cohen's \(f^2 = 0.012137120\)),
contrasting with recent studies. Although the hypothesis is reasonable
and aligns with evolutionary theories of temporal biological systems,
the results suggest that the phenomenon of entrainment is more complex
than previously thought.

%:::% vernacular-abstract body end %:::%

%:::% vernacular-abstract keywords begin %:::%
\hyphenpenalty=100000
\begin{tabular}{p{2.3cm} p{13.6cm}}
  \textbf{Keywords}: & Complexity science. Complex systems. Chronobiology. Biological rhythms. Chronotypes. Circadian phenotypes. Sleep. Entrainment. Latitude. MCTQ.
\end{tabular}
%:::% vernacular-abstract keywords end %:::%
\end{resumoenv}
%:::% vernacular-abstract end %:::%

% -----
% Abstract in the foreign language (mandatory)
% -----

%:::% foreign-abstract begin %:::%
\begin{resumoenv}[\resumoestrangeironame]
\begin{otherlanguage*}{brazil}
%:::% foreign-abstract reference begin %:::%
Vartanian, D. ({\imprimirdata}). \textit{A latitude está associada ao cronotipo?} [Dissertação de Mestrado, Universidade de São Paulo].
%:::% foreign-abstract reference end %:::%

%:::% foreign-abstract body begin %:::%

As teorias sobre ritmos circadianos estão bem estabelecidas na ciência,
mas ainda há a necessidade de testá-las em amostras mais amplas para
compreender melhor a expressão dos fenótipos temporais. Esta dissertação
investiga a hipótese de que a latitude influencia a expressão dos
cronotipos, baseada na ideia de que regiões próximas aos polos recebem
menos luz solar ao longo do ano do que as regiões equatoriais. Esse
diferencial sugere que áreas equatoriais possuem um \emph{zeitgeber}
solar mais forte, o que poderia levar a uma maior sincronização dos
ritmos circadianos com o ciclo claro-escuro, reduzindo a amplitude e a
diversidade de fenótipos circadianos, resultando em uma propensão maior
ao cronotipo matutino. Para testar essa hipótese, foram analisados dados
de \(65.824\) indivíduos de todas as regiões do Brasil, coletados em
2017 com base no Munich ChronoType Questionnaire (MCTQ). A análise,
utilizando modelos de regressão linear aninhados, revelou um efeito
negligenciável da latitude na variação da expressão dos cronotipos
(\(f^2\) de Cohen \(= 0.012137120\)), em contraste com estudos recentes.
Embora a hipótese faça sentido e esteja alinhada com teorias evolutivas
dos sistemas biológicos temporais, os resultados sugerem que o fenômeno
de \emph{entraiment} é mais complexo do que se imagina.

%:::% foreign-abstract body end %:::%

%:::% foreign-abstract keywords begin %:::%
\hyphenpenalty=100000
\begin{tabular}{p{3.6cm} p{12.3cm}}
  \textbf{Palavras-chaves}: &  Ciência da complexidade. Sistemas complexos. Cronobiologia. Ritmos biológicos. Cronotipos. Fenótipos circadianos. Sono. Entrainment. Latitude. MCTQ.
\end{tabular}
%:::% foreign-abstract keywords end %:::%
\end{otherlanguage*}
\end{resumoenv}
%:::% foreign-abstract end %:::%

% -----
% Table of contents (mandatory)
% -----

%:::% table-of-contents begin %:::%
\pdfbookmark[0]{\contentsname}{toc}
\tableofcontents*
\cleardoublepage
%:::% table-of-contents end %:::%
