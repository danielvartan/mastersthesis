%:::% class attribute begin/end %:::%

% -----
% Cover (mandatory)
% -----

%:::% cover begin %:::%
\imprimircapa
%:::% cover end %:::%

% -----
% Title page (mandatory)
% -----

%:::% approval-sheet begin %:::%
\imprimirfolhaderosto
%:::% approval-sheet end %:::%

% -----
% Cataloging record (mandatory)
% -----

%:::% cataloging-record begin %:::%
\begin{fichacatalografica}
\hyphenpenalty=100000
%:::% cataloging-record body begin %:::%
I authorize the full or partial reproduction of this work by any conventional or electronic means for the purposes of study and research, provided that the source is cited.
\small


\vfill
\begin{center}
CATALOGUING IN PUBLICATION

(University of São Paulo. School of Arts, Sciences and Humanities. Library)

{\tiny CRB 8-4936}

\medskip
\ABNTEXfontereduzida
\setlength{\fboxsep}{1cm}
\fbox{
\begin{minipage}[c][6cm]{12cm}
Azevedo, Daniel Kachvartanian de

\hspace{0.5cm} {\imprimirtitulo}  / {\imprimirautor} ; supervisor, {\imprimirorientador}. -- {\imprimirdata}

\hspace{0.5cm} {\thelastpage} p : il.

\smallskip
\hspace{0.5cm} {\imprimirtipotrabalho} (\imprimirtituloacademico) -- {\imprimirprograma}, {\imprimirescola}, {\imprimiruniversidade}.

\hspace{0.5cm} {\imprimirnotadeversao}.

\smallskip
\hspace{0.5cm} 1. Complexity science. 2. Chronobiology. 3. Biological rhythms. 4. Chronotypes. 5. Circadian phenotypes. 6. Entrainment. 7. Latitude. I. Rodrigues Neto, Camilo, supervisor II. Title.

\end{minipage}
}
\end{center}
\vspace{\hugeskipamount}
%:::% cataloging-record body end %:::%
\end{fichacatalografica}
%:::% cataloging-record end %:::%

% -----
% Approval sheet (mandatory)
% -----

%:::% approval-sheet begin %:::%
\begin{folhadeaprovacao}[\folhadeaprovacaoname]
\hyphenpenalty=100000
%:::% approval-sheet body begin %:::%
{\imprimirtipotrabalho} by {\imprimirautor}, under the title \textbf{\imprimirtitulo}, presented to the {\imprimirescola} at the {\imprimiruniversidade}, as a requirement for the degree of {\imprimirtituloacademico} by the {\imprimirprograma}, in the concentration area of {\imprimirareadeconcentracao}.

\vspace{\hugeskipamount}
Approved on February 3, 2025.

\vspace{\hugeskipamount}
\begin{center}
  Examination Committee
\end{center}

\vspace{\smallskipamount}
Committee Chair:

\vspace{\tinyskipamount}
\begingroup

\AtBeginEnvironment{tabular}{
  \normalsize\raggedright
  \renewcommand{\arraystretch}{2}
}

\setlength{\arrayrulewidth}{0pt}
\setlength{\tabcolsep}{0cm}
\begin{tabular}{m{2.5cm} m{13.5cm}}
  Prof. Dr. & Camilo Rodrigues Neto \\
  Institution & School of Arts, Sciences, and Humanities, University of São Paulo \\
\end{tabular}

\vspace{\bigskipamount}
Examiners:

\vspace{\tinyskipamount}
\begin{tabular}{m{2.5cm} m{13.5cm}}
  Prof. Dr. & Tiago Gomes de Andrade \\
  Institution & Faculty of Medicine, Federal University of Alagoas \\
  Evaluation &  Approved \\
\end{tabular}

\vspace{\smallskipamount}
\begin{tabular}{m{2.5cm} m{13.5cm}}
  Prof. Dr. & Domingos Alves \\
  Institution & Ribeirão Preto Medical School, University of São Paulo \\
  Evaluation & Approved \\
\end{tabular}

\vspace{\smallskipamount}
\begin{tabular}{m{2.5cm} m{13.5cm}}
  Prof. Dr. & Marcelo de Souza Lauretto \\
  Institution & School of Arts, Sciences, and Humanities, University of São Paulo \\
  Evaluation & Approved \\
\end{tabular}
\endgroup
%:::% approval-sheet body end %:::%
\end{folhadeaprovacao}
%:::% approval-sheet end %:::%

% -----
% Acknowledgments (optional)
% -----

%:::% acknowledgments begin %:::%
\begin{agradecimentos}[\agradecimentosname]
\hyphenpenalty=100000
%:::% acknowledgments body begin %:::%

This work would not have been possible without the support, love, and
guidance of many remarkable individuals and organizations. I extend my
deepest gratitude to:

\smallskip

My beloved partner, Salete Perroni (Sal), whose unwavering support has
been my constant source of strength. My mother, for her unconditional
love, and my sister and brother, for their love and companionship
throughout life's journey.

My scientific collaborators and dear friends, Alícia Rafaelly Vilefort
Sales and Maria Augusta Medeiros de Andrade. To Professor Humberto
Miguel Garay Malpartida, for his steadfast support, unwavering
principles, and integrity, which shone through when it was most needed.

My supervisor, Professor Camilo Rodrigues Neto, who introduced me to
complexity science in 2012, guided my dissertation with patience, and
demonstrated exceptional virtue in mediating a challenging supervisory
transition. I am equally grateful to Professor Carlos Molina Mendes, who
handled this transition with remarkable speed, impartiality, and
professionalism.

My cherished friends: Alex Azevedo Martins, Augusto Amado, Carina
(Cacau) Prado, Ítalo Alves Bezerra do Nascimento, Júlia Mafra, Letícia
Nery de Figueiredo, Marcelo Ricardo Fernandes Roschel, Reginaldo Noveli,
Sílvia Capelanes, and Vanessa Simon Silva.

Finally, this journey would not have been possible without the
institutional support of USP's Support Program for Student Permanence
and Education (PAPFE) and the Coordination for the Improvement of Higher
Education Personnel (CAPES), whose contributions have been invaluable.

\smallskip
\begingroup
\renewcommand{\baselinestretch}{1}

\noindent This study was financed in part by the Coordenação de
Aperfeiçoamento de Pessoal de Nível Superior - Brasil
(\href{https://www.gov.br/capes/}{CAPES}) - Finance Code 001, Grant
number 88887.703720/2022-00.

\endgroup

%:::% acknowledgments body end %:::%
\end{agradecimentos}
%:::% acknowledgments end %:::%

% -----
% Epigraph (optional)
% -----

%:::% epigraph begin %:::%
\begin{epigrafe}[] % \epigraphname | Keep #1 empty.
\vspace*{\fill} % Don't change it.
\begin{flushright}
%:::% epigraph body begin %:::%
\textit{Nullius in verba}\footnotemark{}

\footnotetext{
  The Royal Society. (n.d.). \textit{History of the Royal Society}. \href{https://royalsociety.org/about-us/history}{https://royalsociety.org/about-us/history}
}
%:::% epigraph body end %:::%
\end{flushright}
\end{epigrafe}
%:::% epigraph end %:::%

% -----
% Abstract in the vernacular language (mandatory)
% -----

%:::% vernacular-abstract begin %:::%
\begin{resumoenv}[\resumoname]
 %:::% vernacular-abstract reference begin %:::%
Vartanian, D. ({\imprimirdata}). \textit{\imprimirtitulo} [{\imprimirtipodetituloacademico}'s {\imprimirtipotrabalho}, {\imprimiruniversidade}].
%:::% vernacular-abstract reference end %:::%

%:::% vernacular-abstract body begin %:::%

Although significant progress has been made in understanding circadian
rhythms, further research with larger and more diverse samples is needed
to deepen our understanding of temporal phenotypes and their
variability. This thesis examines the relationship between latitude and
human chronotype expression, investigating whether variations in annual
sunlight exposure between equatorial and non-equatorial regions
influence circadian phenotypes. The underlying premise suggests that a
stronger solar zeitgeber near the equator should promote greater
entrainment to the light/dark cycle, potentially reducing phenotype
diversity and favoring morningness in equatorial populations. To test
this hypothesis, data from \(65,824\) individuals distributed across a
\(33.85°\) latitude range in Brazil were analyzed. Data collection
employed the Munich ChronoType Questionnaire (MCTQ) during a single
spring week (October 15--21, 2017), minimizing seasonal variations in
photoperiod across regions. The analysis employed nested regression
models weighted according to population proportions at the time of data
collection. Contrary to expectations, results revealed no meaningful
relationship between latitude and chronotype (Cohen's \(f^2 = 0.00308\),
\(95\% \ \text{CI}[0, 0.01214]\)), consistent with recent findings in
the field. All analytical procedures, from raw data processing through
effect size estimation, were conducted using reproducible methods. These
findings contribute to our evidence-based understanding of circadian
rhythm regulation while challenging established assumptions in
chronobiology research. While this study does not refute the hypothesis
outright, the association between latitude and chronotype should remain
an open scientific question rather than settled knowledge until robust
evidence confirms it.

%:::% vernacular-abstract body end %:::%

%:::% vernacular-abstract keywords begin %:::%
\hyphenpenalty=100000
\begin{tabular}{p{2.5cm} p{13.4cm}}
  \textbf{Keywords}: & Complexity science. Complex systems. Chronobiology. Biological rhythms. Chronotypes. Circadian phenotypes. Sleep. Entrainment. Latitude. MCTQ.
\end{tabular}
%:::% vernacular-abstract keywords end %:::%
\end{resumoenv}
%:::% vernacular-abstract end %:::%

% -----
% Abstract in the foreign language (mandatory)
% -----

%:::% foreign-abstract begin %:::%
\begin{resumoenv}[\resumoestrangeironame]
\begin{otherlanguage*}{brazil}
%:::% foreign-abstract reference begin %:::%
Vartanian, D. ({\imprimirdata}). \textit{A latitude está associada ao cronotipo?} [Dissertação de Mestrado, Universidade de São Paulo].
%:::% foreign-abstract reference end %:::%

%:::% foreign-abstract body begin %:::%

Embora avanços significativos tenham sido feitos na compreensão dos
ritmos circadianos, pesquisas adicionais com amostras maiores e mais
diversas são necessárias para aprofundar o entendimento sobre os
fenótipos temporais e sua variabilidade. Esta dissertação examina a
relação entre latitude e a expressão do cronotipo humano, investigando
se variações na exposição anual à luz solar entre regiões equatoriais e
não equatoriais influenciam os fenótipos circadianos. A premissa
subjacente sugere que um \emph{zeitgeber} solar mais forte ao equador
promove uma maior \emph{entrainment} com o ciclo claro/escuro,
potencialmente reduzindo a diversidade fenotípica e favorecendo a
matutinidade em populações equatoriais. Para testar essa hipótese, foram
analisados dados de \(65.824\) indivíduos distribuídos ao longo de um
intervalo latitudinal de \(33,85°\) no Brasil. A coleta de dados foi
realizada com o Munich ChronoType Questionnaire (MCTQ) durante uma única
semana de primavera (15--21 de outubro de 2017), minimizando variações
sazonais no fotoperíodo entre as regiões. A análise empregou modelos de
regressão aninhados ponderados de acordo com as proporções populacionais
no momento da coleta. Contrariando as expectativas, os resultados não
indicaram uma relação significativa entre latitude e cronotipo (\(f^2\)
de Cohen \(= 0,00308\), \(95\% \ \text{IC}[0; 0,01214]\)), em
consonância com achados recentes da área. Todos os procedimentos
analíticos, desde os dados brutos até a estimativa do tamanho do efeito,
foram conduzidos por meio de métodos totalmente reprodutíveis. Esses
achados contribuem para uma compreensão baseada em evidências da
regulação dos ritmos circadianos, ao mesmo tempo que desafiam
pressupostos estabelecidos na pesquisa em cronobiologia. Ainda que este
estudo não refute completamente a hipótese, a associação entre latitude
e cronotipo deve permanecer uma questão científica em aberto, em vez de
ser considerada um conhecimento consolidado, até que evidências robustas
a confirmem.

%:::% foreign-abstract body end %:::%

%:::% foreign-abstract keywords begin %:::%
\hyphenpenalty=100000
\begin{tabular}{p{4cm} p{11.9cm}}
  \textbf{Palavras-chaves}: &  Ciência da complexidade. Sistemas complexos. Cronobiologia. Ritmos biológicos. Cronotipos. Fenótipos circadianos. Sono. Entrainment. Latitude. MCTQ.
\end{tabular}
%:::% foreign-abstract keywords end %:::%
\end{otherlanguage*}
\end{resumoenv}
%:::% foreign-abstract end %:::%

% -----
% Table of contents (mandatory)
% -----

%:::% table-of-contents begin %:::%
\pdfbookmark[0]{\contentsname}{toc}
\tableofcontents*
\cleardoublepage
%:::% table-of-contents end %:::%

% -----
% Other additions
% -----

%:::% other-before-body begin %:::%
%:::% other-before-body end %:::%
